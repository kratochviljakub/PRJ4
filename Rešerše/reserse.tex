\documentclass[12pt,a4paper]{article}
\usepackage{graphicx}
\usepackage[czech]{babel}
\usepackage[utf8]{inputenc}
\usepackage{titling}
\usepackage{pdfpages}
\usepackage[nopar]{lipsum}
\usepackage{mathtools}
\usepackage{multirow}
\usepackage{caption}
\usepackage{float}
\usepackage{enumitem}
\usepackage{listings}
\usepackage{amsmath}
\usepackage{amssymb}
\usepackage[left=25mm,right=25mm,top=30mm,bottom=20mm]{geometry}
\bibliography{clanky}
\bibliographystyle{abbrv}




\begin{document}
\title{SLAM\\rešerše}
\author{Jakub Kratochvíl}
\date{Akademický rok 2017/2018}
\begin{titlepage}
\begin{center}
\includegraphics[scale=0.5]{logo_zcu}\\
\vspace{5cm}
\begin{Large}
\textbf{\thetitle}\\
\end{Large}
\vspace{3cm}
\theauthor\\
\vspace{5cm}
\thedate
\end{center}
\end{titlepage}
\newpage
		
		
\tableofcontents
\newpage
\fontsize{12pt}{18pt}\selectfont


\section{Úvod}
Používaný termín SLAM je zkratka pro simultánní lokalizaci a mapování, což je jeden ze základních problémů autonomních robotů. Jeho řešením by měl být robot, schopný na neznámém místě v neznámém prostředí vytvořit mapu tohoto prostředí a zároveň se v ní sám během pohybu lokalizovat.

Zrod problému SLAM nastal v roce 1986 na konferenci IEEE Robotics and Automation v San Francisku. To bylo období, kdy se teprve začali jak v robotice, tak v umělé inteligenci objevovat pravděpodobnostní metody. Výsledkem diskuzí bylo, že konzistentní pravděpodobnostní mapování se stalo základním problémem robotiky a během následujících let vzniklo několik klíčových prací, které založily statistický základ pro popis vztahů mezi landmarky (orientačními body) a manipulací s geometrickou nejistotou. Klíčové bylo zjištění, že mezi odhady polohy různých landmarků na mapě musí existovat vysoká míra korelace, která roste s následujícími pozorováními. Do té doby se většina výzkumníků snažila korelaci minimalizovat, jenže naopak čím více korelace roste, tím přesnějších výsledků můžeme dosáhnout.

Aplikací SLAM můžeme najít mnoho, počínaje autonomním domácím vysavačem nebo sekačkou na trávu přes robotický průzkum opuštěných nebo člověku nebezpečných prostor, navigaci ponorek kolem podmořských přírodních překážek, řízení bezpilotních letounů a dronů až po v poslední době hodně diskutované samořídící automobily nebo dokonce planetární rovery brázdící povrch Marsu.

\textbf{TODO: předpokládané výsledky této práce}

\newpage
\section{SLAM}
\textbf{TODO: \\
-problém slepice-vejce \\
-metody řešení \\
-je SLAM vyřešen?}

\end{document}